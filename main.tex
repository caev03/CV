%%%%%%%%%%%%%%%%% PREAMBLE %%%%%%%%%%%%%%%%%%%%%%%%%%%%
%Change the font size of your document - 10pt, 12.1pt, etc.
\documentclass[letterpaper,11pt,oneside]{article}
\usepackage[utf8]{inputenc}
\usepackage{setspace}
\usepackage{hyperref}
\usepackage{stmaryrd}

\usepackage{graphicx}
\graphicspath{ {images/}} %upload your signature to this file
%Change the margins to fit your CV/resume content
\usepackage[left=1in, right=1in, bottom=1.25in, top=1.25in]{geometry}

%Skype information - include your Skype name for a link to add you on Skype
\newcommand*{\Skype}{\href{skype:caev03?add}{caev03}} 
\newcommand{\Absender}[1][\normalsize]{\Skype} 

%Changes the page numbers - {arabic}=arabic numerals, {gobble}=no page numbers, {roman}=Roman numerals
\pagenumbering{gobble}

%%%%%%%%%%%%%%%%% END OF PREAMBLE %%%%%%%%%%%%%%%%%%%%%

\begin{document}

%%%%%%%%%%%%%%%%% NAME OF APPLICANT %%%%%%%%%%%%%%%%%%%

\noindent \Huge{\textbf{Camilo ESCOBAR-VEL\'{A}SQUEZ}} \\
\vspace{-2ex}
\hrule 
\normalsize

%%%%%%%%%%%%%%%%% CONTACT INFORMATION %%%%%%%%%%%%%%%%%
% Your email address, website, and Skype name are links to send email, open your website and add you on Skype. 

\begin{center}
\begin{tabular}{ll}
 Universidad de los Andes, Bogota, Colombia 111711   & \hspace{0.5in} \href{mailto:ca.escobar2434@uniandes.edu.co}{ca.escobar2434@uniandes.edu.co} \\
 Systems and Computing Engineering Department    & \hspace{0.5in}  \href{https://caev03.github.io}{https://caev03.github.io}   \\
 Carrera 1 Nº 18A - 12 --	Office: ML-753           & \hspace{0.5in} Skype: \Absender  \\
\end{tabular}
\end{center}

\vspace{2em}

%%%%%%%%%%%%%%%%% MAIN BODY %%%%%%%%%%%%%%%%%%%%%%%%%%%
% The main body is contained in a tabular environment. To move sections onto the next page, simply end the tabular environment and begin a new tabular environment.
\noindent \Large{\textbf{Research Interests}} \\
\vspace{-2ex}
\hrule 
\normalsize
\vspace{2ex}
\noindent \textbf{Software Engineering}, evolution and maintenance, repository mining, \textbf{Software Testing}, automated/automatic testing of Android Apps, automated testing of Web Apps, \textbf{Visual Analytics}, software understandability, \textbf{Mobile Health} \\
\\

\noindent \Large{\textbf{Education}} \\
\vspace{-2ex}
\hrule 
\normalsize
\vspace{2ex}
\noindent \textbf{Ph.D. Engineering, 2019 - 2023} \\
	Universidad de los Andes, Colombia \\
	{\small \textit{Thesis Title:} On improving analysis and testing of open- and closed-source Android Apps \\}
	\textit{Advisor:} Mario Linares-V\'{a}squez, Universidad de los Andes, Colombia \\
	\textit{Co-Advisor:} Gabriele Bavota, Università della Svizzera italiana, Switzerland \\
	\textit{Commitee:} Denys Poshyvanyk, Massimiliano Di Penta, Nicol\'as Cardoso\\
	\\
\textbf{Ph.D. Research Fellowship, Informatics, Sept. 2020 - Aug. 2021.} \\
	\textbf{Swiss Government Excellence Scholarship for Foreign Scholars} \\
	Università della Svizzera italiana \\
	Field: Automation of Software Engineering Tasks \\
	\textit{Advisor:} Gabriele Bavota, Università della Svizzera italiana, Switzerland \\
	\\
\textbf{M.Sc. Software Engineering, 2017 - 2018.} \\
	Universidad de los Andes, Colombia  \\
	\textit{Advisor:} Mario Linares-V\'{a}squez \\
	{\small \textit{Thesis Title:} Automatic Analysis of Android Closed-Source apps to Support Software Engineering Tasks: The Mutation Testing Case \\}
	\\
\textbf{B.S. Systems and Computing Engineering, 2012 - 2016.} \\
	Universidad de los Andes, Colombia  \\
	Minor: Mathematics \\
	\textit{Advisor:} Jose Tiberio Hern\'{a}ndez \\
	
\newpage
\noindent \Large{\textbf{Research Experience}} \\
\vspace{-2ex}
\hrule 
\normalsize
\vspace{2ex}
\noindent \indent \textbf{Professional Researcher} - January 2022 - June 2023\\
\indent{\small Ministery of Science; Fundaci\'on CardioInfantil; Universidad de los Andes, Colombia} \\
\noindent \indent \textbf{Graduate Doctoral Teaching Assistant} - Jan. 2019 - December. 2021\\
\indent{\small Systems and Computing Engineering Department - Universidad de los Andes, Colombia} \\
\noindent \indent \textbf{Graduate Research Assistant} - Jan. 2017 - Dec. 2018\\
\indent{\small Systems and Computing Engineering Department - Universidad de los Andes, Colombia } \\
\\
\noindent \Large{\textbf{Teaching Experience}} \\
\vspace{-2ex}
\hrule 
\normalsize
\vspace{2ex}
\noindent \large{\textbf{Assistant Professor}} \\
\indent \textbf{Mobile App Development} -  August 2023 - Present\\
\indent Systems and Computing Engineering - Universidad de los Andes, Colombia \\
\indent \textbf{Web Development} -  August 2023 - Present \\
\indent Systems and Computing Engineering - Universidad de los Andes, Colombia \\
\indent \textbf{Automated Software Testing} -  August 2023 - Present\\
\indent Masters in Software Engineering - Coursera - Universidad de los Andes, Colombia \\
\indent \textbf{Software Engineering for Mobile Apps} -  August 2023 - Present \\
\indent Masters in Software Engineering - Coursera - Universidad de los Andes, Colombia \vspace{0.2cm}\\ 
\noindent \large{\textbf{Lecturer}} \\
\indent \textbf{Mobile App Development} -  January 2023 - June 2023\\
\indent Systems and Computing Engineering - Universidad de los Andes, Colombia \\
\indent \textbf{Web Development} -  January 2023 - June 2023\\
\indent Systems and Computing Engineering - Universidad de los Andes, Colombia \\
\indent \textbf{Automated Software Testing} -  January 2021 - June 2023\\
 \indent Masters in Software Engineering - Coursera - Universidad de los Andes, Colombia \\
 \indent \textbf{Software Engineering for Mobile Apps} -  January 2023 - June 2023\\
 \indent Masters in Software Engineering - Coursera - Universidad de los Andes, Colombia \vspace{0.2cm}\\ 
 \noindent \large{\textbf{Teacher Assistant}}\\
\indent \textbf{Software Engineering for Mobile Apps} - June 2021 - December 2022  \\
\indent Masters in Software Engineering - Coursera - Universidad de los Andes, Colombia \\
\indent \textbf{Automated Software Testing} - January 2018 - December 2020 \\
\indent Masters in Software Engineering - Universidad de los Andes, Colombia \\
\indent \textbf{Mobile App Development} - January 2020 - December 2020 \\
\indent Systems and Computing Engineering - Universidad de los Andes, Colombia \vspace{0.2cm}\\
\noindent \large{\textbf{Lab Instructor}}\\
\indent \textbf{Design and Analysis of Algorithms}, \textit{Lab Instructor} - Jan. 2017 - Jun. 2017 \\
\indent Systems and Computing Engineering - Universidad de los Andes, Colombia \\

\noindent \LARGE{\textbf{Professional Service}} \\
\vspace{-2ex}
\hrule 
\normalsize
\vspace{2ex}
\noindent \large{\textbf{Organizing Comitee}}\\
\indent\textbf{The 39th International Conference on Software Maintenance and Evolution}\\
\indent ICSME'23, Bogotá, Colombia. \href{https://conf.researchr.org/home/icsme-2023}{https://conf.researchr.org/home/icsme-2023} \\
\indent\textbf{The 39th International Conference on Software Maintenance and Evolution}\\
\indent ICSME'23, Bogotá, Colombia. \href{https://conf.researchr.org/home/icsme-2023}{https://conf.researchr.org/home/icsme-2023} \\
\indent\textbf{The 1st Languages and Tools for Next Generation Testing Workshop} \\
\indent LANGETI'20, Nov 9, 2020, Sacramento, CA, USA. \href{https://langeti.github.io/}{https://langeti.github.io/} \\
\\
\noindent \large{\textbf{Program Comitee}} \\
\indent \textbf{ICPC'24}. TBD, 2024, Lisboa, Portugal. \textit{PC - Research Track} \\
\indent \textbf{MSR'23}. May 15-16, 2023, Melbourne, Australia. \textit{Junior PC - Research Track} \\
\indent \textbf{MOBILESoft'22}. May 22-23, 2022, Pittsburgh, PA, USA. \textit{Tool Demo and Mobile Apps.} \\
\indent \textbf{ESEC/FSE'21}. August 18-29, 2021, (Virtually). \textit{Artifacts Track.} \\
\indent \textbf{MSR'21}. May 23-24, 2021 (Virtually). \textit{Shadow PC - Research Track} \\
\\
\noindent \large{\textbf{Student Volunteer}} \\
\indent \textbf{ICST'20}, Oct 24-28, 2020, Porto, Portugal. \\
\indent \textbf{ICSME'20}, Sep 27 - Oct 3, 2020, Adelaide, Australia. \\
\indent \textbf{ICSE'20}, July 6-11, 2020, Seoul, South Korea. \\
\indent \textbf{ASE'19}, November 11-15, 2019 - San Diego, CA, USA. \\
\indent \textbf{ICSE'19} - May 25-31, 2019 - Montreal, Canada. \\
\indent \textbf{ICSME'18} - September 23-29, 2018 - Madrid, Spain. \\
\noindent \large{\textbf{Talks}} \\
\indent {\textit{\textbf{On improving quality of analysis and testing for Open- and Closed-source Android App}}. Encuentro de Experiencias de Investigaci\'on en Ingeniería EEII2022 at Universidad de los Andes - December 2022 - Bogot\'a, Colombia}
\\
\indent \textit{\textbf{Enabling Automated Software Engineering Tasks for Closed-Source Android Apps}}. 70th Anniversary of Research and Creation at Universidad de los Andes - November 2018 - Bogot\'a, Colombia
\\ \newpage
\noindent \Large{\textbf{Honors and Awards}} \\
\vspace{-2ex}
\hrule 
\normalsize
\vspace{2ex}
\noindent \indent \textbf{Scholarship from Ministry of Science } \\
\indent Scholarship provided to PhD Students - Minciencias 909\\
\indent Jan. 2022 - Dec. 2022 - Bogot\'a, Colombia \\
\\
\noindent \indent \textbf{Swiss Government Excellence Scholarships for Foreign Scholars} \\
\indent \textit{Enabling Automated Software Engineering Tasks for Closed-Source Mobile Apps}\\
\indent Sep. 2020 - Aug. 2021 - Lugano, Switzerland \\
\\
\noindent \indent \textbf{Google Latin American Research Award} \\
\indent \textit{Enabling Automated Software Engineering Tasks for Closed-Source Mobile Apps}\\
\indent 2020 - 2021 - Bello Horizonte, Brasil \\
\\
\noindent \indent \textbf{Google Latin American Research Award} \\
\indent \textit{Enabling Automated Software Engineering Tasks for Closed-Source Mobile Apps}\\
\indent 2019 - 2020 - Bello Horizonte, Brasil \\
\\
\noindent \indent \textbf{ACM SIGAI travel scholarship} \\
\indent \textit{The 34th International Conference on Automated Software Engineering }\\
\indent 2019 - San Diego, CA, USA \\
\\
\noindent \indent \textbf{Google Latin American Research Award} \\
\indent \textit{Enabling Automated Software Engineering Tasks for Closed-Source Mobile Apps}\\
\indent 2018 - 2019 - Bello Horizonte, Brasil \\
\\
\noindent \indent \textbf{1st Place Team Competition} \\
\indent Testathon - Chapter 30\\
\indent April 2019 - Bogotá, Colombia \\
\\
\noindent \indent \textbf{1st Place Hackaton Against Human Trafficking} \\
\indent FIND: A Technological Environment to fight Human Trafficking\\
\indent November 2018 - Bogotá, Colombia \\
\\ \newpage
\noindent \Large{\textbf{Journal Publications}} \\
\vspace{-2ex}
\hrule 
\normalsize
\vspace{0.5cm}

\noindent \textbf{[J5]} Ravelo-Méndez, W., \textbf{Escobar-Velásquez, C.},  Linares-Vásquez, M., \textit{``Kraken 2.0: A platform-agnostic and cross-device interaction testing tool."}, in Science of Computer Programming 2022\\
\\
\noindent \textbf{[J4]} Mazuera-Rozo, A., \textbf{Escobar-Velásquez, C.}, Espitia-Acero, J., Vega-Guzm\'an, D., Trubiani, C., Linares-V\'asquez, M., Bavota, G., \textit{``Taxonomy of security weaknesses in Java and Kotlin Android apps"}, in Journal of Systems and Software 2022 (JSS'22)\\
\\
\noindent \textbf{[J3] Escobar-Velásquez, C.}, Mazuera-Rozo, A., Bedoya, C., Osorio-Riaño, M., Linares-Vásquez, M., Bavota, G., \textit{``Studying eventual connectivity issues in Android apps"}, in Empirical Software Engineering 2022 (EMSE'22)\\
\\
\noindent \textbf{[J2]} Ravelo-Méndez, W., \textbf{Escobar-Velásquez, C.},  Linares-Vásquez, M., \textit{``Kraken: A framework for enabling multi-device interaction-based testing of Android apps."}, in Science of Computer Programming 2021\\
\\
\noindent \textbf{[J1] Escobar-Velásquez, C.}, Linares-V\'asquez, M., Bavota, G., Tufano, M., Moran, K., Di penta, M., Vendome, C., Bernal-Cardenas, C., Poshyvanyk, D., \textit{``Enabling Mutant Generation for Open- and Closed-Source Android Apps"}, in IEEE Transactions on Software Engineering 2020 (TSE’20)\\
\\
\noindent \Large{\textbf{Conference Publications}} \\
\vspace{-2ex}
\hrule 
\normalsize
\vspace{2ex}
\noindent \textbf{[C13]} Mazuera-Rozo, A., \textbf{Escobar-Velásquez, C.}, Espitia-Acero, J., Linares-V\'asquez, M., Bavota, G., \textit{``Detecting connectivity issues in android apps"}, in Proceedings of the 29th IEEE International Conference on Software Analysis, Evolution and Reengineering (SANER'22) - Research Track, Honolulu, HI, USA, March 15-18, 2020. \\
\\
\noindent \textbf{[C12]} Ravelo-Méndez, W., \textbf{Escobar-Velásquez, C.},  Linares-Vásquez, M., \textit{``Kraken 2.0: A platform-agnostic and cross-device interaction testing tool"}, in Proceedings of the 29th IEEE International Conference on Software Analysis, Evolution and Reengineering (SANER'22) - Tool Demo Track, Honolulu, HI, USA, March 15-18, 2020. \\
\\
\noindent \textbf{[C11]} Naranjo-Puentes, S., \textbf{Escobar-Velásquez, C.}, Vendome, C., Linares-V\'asquez,  M., \textit{``A Preliminary Study on Accessibility of Augmented Reality Features in Mobile Apps"}, in Proceedings of the 29th IEEE International Conference on Software Analysis, Evolution and Reengineering (SANER'22) - Early Research Achievement Track, Honolulu, HI, USA, March 15-18, 2020. \\
\\
\noindent \textbf{[C10]} Mojica-Hanke, A., Bello-Jiménez, L., \textbf{Escobar-Velásquez, C.}, Linares-V\'asquez,  M., \textit{``Cr\`eme de la cr\`eme. Investigating Metadata and Survivability of Top Android Apps"}, in Proceedings of the 29th IEEE International Conference on Software Analysis, Evolution and Reengineering (SANER'22) - REproducibility Studies and NEgative Results Track, Honolulu, HI, USA, March 15-18, 2020. \\
\\
\noindent \textbf{[C9] Escobar-Velásquez, C.}, Donoso-Diaz, A., Linares-V\'asquez,  M., \textit{``ITDroid: A Tool for Automated Detection of i18n Issues on Android Apps"}, in Proceedings of the 8th International Conference on Mobile Software Engineering and Systems 2021 (MobileSoft'21)) - Tool Demo Track, MAdrid, Spain, May 17 - 19, 2021. \\
\\
\noindent \textbf{[C8] Escobar-Velásquez, C.}, Riveros, D., Linares-V\'asquez,  M., \textit{``Mutapk 2.0: a tool for reducing mutation testing effort of android apps"}, in Proceedings of the 28th ACM Joint Meeting on European Software Engineering Conference and Symposium on the Foundations of Software Engineering 2020 (ESEC/FSE’20) - Tool Demo Track, Sacramento, CA, USA, November 6 - 16, 2020. \\
\\
\noindent \textbf{[C7] Escobar-Velásquez, C.}, Osorio-Ria\~no, M., Dominguez-Osorio, J., Arevalo, M., Linares-V\'asquez,  M., \textit{``An empirical study of i18n collateral changes and bugs in guis of android apps"}, in Proceedings of the 36th IEEE International Conference on Software Maintenance and Evolution 2020 (ICSME’20) - Research Track, Adelaide, Australia, September 27th - October 3rd, 2020. \\
\\
\noindent \textbf{[C6]} Bello-Jiménez, L., \textbf{Escobar-Velásquez, C.}, Mojica-Hanke, A., Linares-V\'asquez,  M., \textit{``Hall-of-Apps: The Top Android Apps Metadata Archive"}, in Proceedings of the 17th International Conference on Mining Software Repositories 2020 (MSR’20) - Data Showcase Track, Seoul, South Korea, October 5-11, 2020. \\
\\
\noindent \textbf{[C5]} Trujillo, M., Linares-V\'asquez,  M., \textbf{Escobar-Velásquez, C.}, Dusparic, I., Cardozo, N., \textit{``Does Neuron Coverage Matter for Deep Reinforcement Learning? A Preliminary Study"}, in Proceedings of the 2nd Workshop on Testing for Deep Learning and Deep Learning for Testing 2020 (DeepTest’20) - Research Track, Seoul, South Korea, October 5-11, 2020. \\\\
\noindent \textbf{[C4] Escobar-Velásquez, C.}, \textit{``Source-Codeless Testing for Android Apps"}, in Proceedings of the 13th IEEE International Conference on Software Testing, Validation and Verification 2020 (ICST’20), Doctoral Symposyum, Porto, Portugal, October 24-28, 2020. \\\\
\noindent \textbf{[C3] Escobar-Velásquez, C.}, Osorio-Riaño, M., Linares-Vásquez, M., \textit{``MutAPK: Source-Codeless Mutant Generation for Android Apps"}, in Proceedings of the 34th International Conference on Automated Software Engineering 2019 (ASE’19), Tool Demo, San Diego, USA, November 11-15, 2019. \\\\
\noindent \textbf{[C2]} Ravelo-Méndez, W., \textbf{Escobar-Velásquez, C.}, Linares-Vásquez, M., \textit{``Kraken-Mobile: Cross-Device Interaction-based Testing of Android Apps"}, in Proceedings of the 35th International Conference on Software Maintenance and Evolution 2019 (ICMSE’19), Tool Demo, Cleveland, OH, United States, October 2-4, 2019. \\\\
\noindent \textbf{[C1]} Linares-Vásquez, M., Bavota, G., \textbf{Escobar-Velásquez, C.}, \textit{``An Empirical Study on Android-Related Vulnerabilities"}, in Proceedings of the 14th International Conference on Mining Software Repositories 2017 (MSR ’17), Research Paper, Buenos Aires, Argentina, May 20-21, 2017. \\

%
%\noindent \Large{\textbf{Publications under Review and Submission}} \\
%\vspace{-2ex}
%\hrule 
%\normalsize
%\vspace{2ex}
%\noindent \textbf{Escobar-Velásquez, C.}, Linares-Vásquez, M., Bavota, G., Tufano, M., Moran, K., Di Penta, M., Vendome, C., Bernal-Cárdenas, C., Poshyvanyk, D., \textit{"CONAN: Statically Detecting Connectivity Issues in Android Applications"}, Under review for the 31st ACM Joint European Software Engineering Conference and Symposium on the Foundations of Software Engineering (ESEC/FSE'23).\\
%\\
\noindent \Large{\textbf{Student Supervision - Undergraduate Projects} \\
\vspace{-2ex}
\hrule 
\normalsize
\vspace{2ex}
\noindent\textbf{2023-II}\\
\indent\textbf{Esteban Martinez-Hernandez.} {\footnotesize Topic: \textit{Orchestration of IoT procedures in educational environments}}\\
\indent\textbf{Felipe Duque-Martinez.} {\footnotesize Topic: \textit{Visual Analytics}}\\
\textbf{Camila Pantoja-G\'omez. Universidad de los Andes, Colombia. 2022-I.} \\
\noindent\textbf{Daniel Valbuena-Bautista. Universidad de los Andes, Colombia. 2022-II.} \\
\indent {\footnotesize Project Title: \textit{Automatic GUI testing for android using reinforcement learning}}\\
\textbf{Camila Pantoja-G\'omez. Universidad de los Andes, Colombia. 2022-I.} \\
\noindent\textbf{Daniel Valbuena-Bautista. Universidad de los Andes, Colombia. 2022-II.} \\
\indent {\footnotesize Project Title: \textit{Automatic GUI testing for android using reinforcement learning}}\\
\textbf{Camila Pantoja-G\'omez. Universidad de los Andes, Colombia. 2022-I.} \\
\indent {\footnotesize Project Title: \textit{Automatic Multi-platform Interaction Testing for Android Using Reinforcement Learning}}\\
\textbf{Edgar D\'iaz-Su\'arez. Universidad de los Andes, Colombia. 2022-I.} \\
\indent {\footnotesize Project Title: \textit{Automatic Multi-platform Interaction Testing for Android Using Reinforcement Learning}}\\
\textbf{Camilo Rozo-Benitez. Universidad de los Andes, Colombia. 2022-I.} \\
\indent {\footnotesize Project Title: \textit{Automatic Multi-platform Interaction Testing for Android Using Reinforcement Learning}}\\
\textbf{Mario Andrade-Vargas. Universidad de los Andes, Colombia. 2022-I.} \\
\indent {\footnotesize Project Title: \textit{Towards automated repairment of internationalization issues for android apps}}\\
\textbf{Andr\'es Donoso-D\'iaz. Universidad de los Andes, Colombia. 2021-II.} \\
\indent {\footnotesize Project Title: \textit{Improving Automated i18n Testing of Android Apps}}\\
\textbf{Michael Osorio-Ria\~no. Universidad de los Andes, Colombia. 2020-I.} \\
\indent {\footnotesize Project Title: \textit{Comparison and analysis between automatic exploration tools for Android applications}}\\
\textbf{Juan Dom\'inguez-Osorio. Universidad de los Andes, Colombia. 2019-II.} \\
\indent {\footnotesize Project Title: \textit{Automated GUI ripping for web applications}}\\
\textbf{Andr\'es L\'opez-Fabara. Universidad de los Andes, Colombia. 2019-II.} \\
\indent {\footnotesize Project Title: \textit{Automated GUI ripping for web applications}}\\
\textbf{William Ravelo-M\'endez. Universidad de los Andes, Colombia. 2019-I.} \\
\indent {\footnotesize Project Title: \textit{Automated cross-app testing}}\\

\end{document}

